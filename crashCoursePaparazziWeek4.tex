\documentclass{article}

\usepackage[margin=3cm]{geometry}

\usepackage{fancyhdr}
\pagestyle{fancy}
\usepackage{graphicx}
\usepackage{color}
\usepackage{xcolor}
\usepackage{hyperref}
%% Define no indent with line skip with new paragraph
\usepackage{parskip}

\definecolor{code}{rgb}{.92,.92,.99}
\definecolor{codered}{rgb}{.92,.62,.62}
\definecolor{darkgreen}{rgb}{.2,.62,.18}
\usepackage{listings}
\lstset{xrightmargin=10pt,xleftmargin=10pt,language=Java,captionpos=b,tabsize=3,frame=none,keywordstyle=\color{blue},commentstyle=\color{darkgreen},stringstyle=\color{red},showstringspaces=false,basicstyle=\footnotesize\ttfamily,emph={label},backgroundcolor=\color{code}}
%%numbers=left,numberstyle=\tiny,numbersep=5pt,breaklines=true,\

\usepackage{soul}
\lstdefinestyle{Bash}
{language=bash,
  escapechar=!,
  backgroundcolor=\color{white},
  columns=fullflexible,
  breaklines=true,         
  breakatwhitespace=false,
  inputencoding=utf8x,
  keywordstyle=\color{black},
  basicstyle=\small\ttfamily,
%  morekeywords={peter@kbpet},
%  alsoletter={:~$},
%  morekeywords=[2]{peter@kbpet:},
%  keywordstyle=[2]{\color{red}},
%  literate={\$}{{\textcolor{red}{\$}}}1 
%         {:}{{\textcolor{red}{:}}}1
%         {~}{{\textcolor{red}{\textasciitilde}}}1,
  literate={~}{\raisebox{-0.5ex}{\textasciitilde}}1,
}

\newenvironment{code}
{\begin{minipage}[l]{\textwidth}}
{\end{minipage}}

\newcommand{\ccpmaketitle}[4][Roland Meertens, Christophe de Wagter, Guido de Croon, Tom van Dijk] {
	\fancyhead[LO,LE]{Crash Course Paparazzi 2019}
	\fancyfoot[LO,LE]{{\small Crash Course Paparazzi 2019}}
	\fancyfoot[CO,CE]{\thepage}
	\author{#1}
	\title{\bf Crash Course Paparazzi 2019\\#2\\{\large #4#3}}
	\date{February 2019}
	\setlength{\parindent}{0em}
	\maketitle
}


\begin{document}
\ccjmaketitle{Computer vision}{\ldots Making your drone see something}{lesson 4}

\subsection*{Introduction}


\subsection*{Goals of this exercise}
\begin{itemize}
\item Finding the camera data
\item The format of the image
\item First small projects
\end{itemize}

\subsection*{Finding the camera data}
To get data out of the camera in paparazzi you have to include the video\_thread.xml module in your airframe file, and define VIDEO\_THREAD\_CAMERA to front\_camera.
After you added this you can create a new module that does something with the data of the front camera. A nice example is the colorfilter module. Please take some time to check what the colorfilter does. 
In the colorfilter\_init function the function cv\_add is called with as argument the colorfilter\_func function. 
Now each time the video thread gets an image from the camera it calls the colorfilter\_func function with the new image as an argument. Note that the order in which you call cv\_add is important. If you add functions, the first one you added will be called first. As the argument the function gets is a pointer to an image it is possible for a module to change the actual image. The colorfilter also does this, if you add both a colorfilter that filters on orange, and then a colorfilter that filters on red, the second colorfilter will never see red. 

\subsection*{The format of the image}

\subsection*{First steps}
Nice first exercises are:
\subsubsection*{Avoid orange}
 keep flying forward untill you detect a lot of orange pixels in your image. Then hover at that place. 
Try this by letting the drone fly through the arena and suddenly putting an orange pole in front of its forward camera.

\subsubsection*{Read a sign}
Fly to a position in the center of the arena. If you see a blue sign fly to a waypoint to the left for five seconds, if you see a red sign fly to a waypoint to the right for five seconds. 

\subsubsection*{Follow me!}
Create a real "selfie-drone" in which the camera follows a red blob. 
Start by hovering at a waypoint and determining where the red blob is (more to the left or more to the right?) and adjust your heading based on this. 
To start following you you can determine the size of the blob. Is it too big? Fall back! Otherwise: fly slowly forwards.
\end{document}
