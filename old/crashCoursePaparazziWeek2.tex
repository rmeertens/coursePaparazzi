\documentclass{article}

\usepackage[margin=3cm]{geometry}

\usepackage{fancyhdr}
\pagestyle{fancy}
\usepackage{graphicx}
\usepackage{color}
\usepackage{xcolor}
\usepackage{hyperref}
%% Define no indent with line skip with new paragraph
\usepackage{parskip}

\definecolor{code}{rgb}{.92,.92,.99}
\definecolor{codered}{rgb}{.92,.62,.62}
\definecolor{darkgreen}{rgb}{.2,.62,.18}
\usepackage{listings}
\lstset{xrightmargin=10pt,xleftmargin=10pt,language=Java,captionpos=b,tabsize=3,frame=none,keywordstyle=\color{blue},commentstyle=\color{darkgreen},stringstyle=\color{red},showstringspaces=false,basicstyle=\footnotesize\ttfamily,emph={label},backgroundcolor=\color{code}}
%%numbers=left,numberstyle=\tiny,numbersep=5pt,breaklines=true,\

\usepackage{soul}
\lstdefinestyle{Bash}
{language=bash,
  escapechar=!,
  backgroundcolor=\color{white},
  columns=fullflexible,
  breaklines=true,         
  breakatwhitespace=false,
  inputencoding=utf8x,
  keywordstyle=\color{black},
  basicstyle=\small\ttfamily,
%  morekeywords={peter@kbpet},
%  alsoletter={:~$},
%  morekeywords=[2]{peter@kbpet:},
%  keywordstyle=[2]{\color{red}},
%  literate={\$}{{\textcolor{red}{\$}}}1 
%         {:}{{\textcolor{red}{:}}}1
%         {~}{{\textcolor{red}{\textasciitilde}}}1,
  literate={~}{\raisebox{-0.5ex}{\textasciitilde}}1,
}

\newenvironment{code}
{\begin{minipage}[l]{\textwidth}}
{\end{minipage}}

\newcommand{\ccpmaketitle}[4][Roland Meertens, Christophe de Wagter, Guido de Croon, Tom van Dijk] {
	\fancyhead[LO,LE]{Crash Course Paparazzi 2019}
	\fancyfoot[LO,LE]{{\small Crash Course Paparazzi 2019}}
	\fancyfoot[CO,CE]{\thepage}
	\author{#1}
	\title{\bf Crash Course Paparazzi 2019\\#2\\{\large #4#3}}
	\date{February 2019}
	\setlength{\parindent}{0em}
	\maketitle
}


\begin{document}
\ccpmaketitle[Tom van Dijk, Roland Meertens, Christophe de Wagter, Guido de Croon]{Your first autonomous flight}{\ldots Your first autonomous flight}{Lesson 2}

\subsection*{Introduction}
This week is a very exciting week: you will have your first manual flight and you will perform your first autonomous flight.

\subsection*{Goals of today}
\begin{itemize}
\item Prepare your first flight
\item Practice your manual flying skills in simulation
\item Upload a program to your drone
\item Fly manually with your drone
\item Simulate your first autonomous flight
\item Get a 3D fix on your drone
\item Perform your first autonomous flight
\item Edit your flightplan
\item Create a safety rule
\end{itemize}

\section*{Downloading and installing Paparazzi}
Paparazzi only runs on Linux.
A bootable USB image is provided on the course website. This image contains a working Linux installation with pre-installed Paparazzi, Eclipse and OpenCV. See the website \url{https://tudelft.github.io/coursePaparazzi/} for more information.

If you want to install Paparazzi on an existing Linux installation (native OS or dual boot), a summary of the installation steps is included at the end of this document.




\section*{Your first flight with Paparazzi}
In this exercise you will familiarize yourself with Paparazzi and the Bebop drone by performing a short manual flight.

\subsection*{Pre-flight preparation}
Before you begin your preparation, charge your battery if you have not already done so. Charging can take a significant amount of time, so you should do this beforehand or during your preparation.

\subsubsection*{Connect and calibrate your joystick}
Before you can fly, you need to connect and calibrate your joystick.
Connect the joystick to your computer. Ensure that the trim sliders to the right and below the two sticks are in the center position.

Open a terminal (Ctrl+Alt+T), type \fbox{jstest-gtk} and press enter to start the joystick calibration program. Find your joystick in the list. If your device is not \verb"/dev/input/js0", write down the number after `js', you will need this later. Double-click on your joystick to start the test and calibration program.

Click the `Calibration' button, then `Start Calibration'. Move both sticks in circles, then return them to the center position (the throttle should also be centered). Switch the autopilot mode switch on the top left between the two positions. Move the tuning knob on the top right fully clockwise and counterclockwise. With the control sticks centered, click on the `Ok' and `Close' buttons.

Joystick calibration should be repeated every time you connect a joystick to your computer, as the center positions may vary slightly per device and the trim sliders may have shifted. It is also a good test to verify that your joystick is working correctly.

\subsubsection*{Start Paparazzi Center}
Start Paparazzi: go back to your terminal and navigate to the paparazzi directory: \fbox{cd paparazzi}. Then, start the Paparazzi Center using \fbox{./paparazzi}.

\medskip
The Paparazzi Center performs the following functions: it manages aircraft, airframes and flightplans, it is used to compile and upload your code and it is used to start flight- or simulation sessions.

The top left corner shows the current aircraft (A/C). Use the drop-down menu to select `bebop\_orange\_avoid'.
Below the A/C there are fields to select your airframe and flight plan and other options. Leave these as they are for now; later in the course you may want to modify your flight plan or airframe.

The buttons in the top center (Target, Flash mode, Clean, Build, Upload) are used to compile and upload your code.

Finally, the top right allows you to execute a `session': a collection of programs that help you perform your flight. Examples of these programs are the communications server and datalink, the Ground Control Station and your joystick.



\subsection*{Simulation}
Before you fly with the real drone, you will first practice in the simulator.
This allows you to practice without the risk of breaking your drone. It also allows you to practice somewhere else than the CyberZoo as long as you have access to a joystick.

\medskip
The simulation is started as follows:
\begin{enumerate}
\item In the Paparazzi Center, go to the Target drop-down menu and select the `nps' target. NPS is the \emph{New Paparazzi Simulator}; by setting nps as the target your autopilot is compiled to run on your own pc instead of on the Bebop.
\item Click `Clean', then `Build'. Paparazzi will now compile your autopilot.
\item In the Session drop-down menu, select `Simulation - Gazebo + Joystick' (\emph{not} `Simulation'). Press execute to start the simulation.\smallskip\\
If your joystick device was not \verb"/dev/input/js0" the following error will appear in the Paparazzi center: \verb|Invalid_argument("index out of bounds")|. Fix this as follows: the command in the `Joystick' textbox ends with \verb"-d 0"; replace the 0 with the number of your joystick device (e.g. if your joystick was \verb"/dev/input/js1", change the command to \verb"... -d 1"). Click the `Stop' button next to the Joystick textbox, then click `Redo'. In the top menu of the Paparazzi Center, click `Session $\rightarrow$ Save' and overwrite the `Simulation - Gazebo + Joystick' session. Paparazzi will now remember your joystick number. You may have to repeat these steps for the `Flight UDP' session.
\end{enumerate}

After these steps, two new windows will have appeared: `GCS' and `Gazebo'.
GCS is the \emph{Ground Control Station}, from here you can control and monitor your drone.
The top left of this screen shows the status of your drone. Pay extra attention to the following fields:
\begin{itemize}
\item Bat: this shows the battery level of your drone. Stop flying when the voltage drops below 11.1, otherwise you will damage the battery. This is of course not a problem in the simulator.
\item Link: indicates that the drone and GCS are connected. This should be green.
\item Status:
\begin{itemize}
\item Autopilot mode (ATT or NAV): this is the current mode of the autopilot. In ATT mode you are in control of the drone, in NAV mode the drone will follow the flight plan. Check that the mode switch on the top left of the joystick toggles the autopilot mode, then set it to ATT.
\item RC/Joystick status (OK): this indicates that your joystick is operating correctly.
\item GPS status (3D): this indicates that the drone has a position fix. This field should be green and at `3D' for autonomous flight.
\end{itemize}
\item Throttle (red, 0\%): this shows the current throttle level. The indicator is red when the motors are killed and orange/green when the motors are running.
\end{itemize}
The right part of the GCS window shows a map of your drone and flight plan (top), and the flight plan itself (bottom). The autonomous flight exercise will show how to use the flight plan.

\medskip
The Gazebo window provides a view of your simulation. Navigate around the Cyberzoo by panning (left mouse button), zooming (scroll) and rotating (middle mouse button).
You can reduce the clutter by going to the `Layers' tab (top left) and unchecking the checkboxes. (Note: the CyberZoo and environment will remain visible to the drone's camera).

\medskip
Now that you are familiar with the GCS and Gazebo, it is time for your first flight.
Open the GCS and make sure that the autopilot is in ATT mode (use the top left switch of the joystick). Then, move the throttle/yaw stick (left) fully to the bottom-right position until the throttle indicator turns orange: you have now started the motors.

Switch to the Gazebo window. Slowly advance the throttle until the drone takes off.
You can now use the simulator to practice your manual flying skills. It is very important to test if you can fly a drone nose-in (with its nose pointing towards you). When you can do this perfectly you can consider yourself good enough to serve as a safety pilot during this course.

When you are finished, you can stop the simulation by going to the Paparazzi Center (not the GCS) and clicking the `Stop/Remove All Processess' button.



\subsection*{Manual flight}
Once you have mastered flying in the simulator, it is time to transition to the real drone.

\begin{enumerate}
\item Place the Bebop inside the CyberZoo. Connect the battery and turn the drone on using the power button on the rear side. When the light stops blinking, press the power button four times. (This allows custom code to be uploaded).
\item Return to your computer and connect to your Bebop's WiFi access point. (Note: disable and re-enable the WiFi in Ubuntu if the Bebop takes too long to appear in the list of available networks.)
Make sure that you have connected to the right Bebop, it is a good idea to write down its number somewhere.
\item In the Paparazzi Center, select the `ap' target and click `Clean' and `Build'. Check that you are connected to the right Bebop, then click the `Upload' button. You should get a message that the autopilot has been started successfully.
\item Select the `Flight UDP' session and click `Execute'.
\end{enumerate}

After these steps, the GCS window should appear.
Ensure that the battery level is sufficient, the Link indicator is green, the RC status indicator is `OK' and that the autopilot is in ATT mode. Since you will fly manually, it is ok if the GPS fix indicator is red.
If everything is in order, start the engines by moving the yaw/throttle stick to the bottom-right. Slowly increase the throttle to take off, then enter a stable hover. Congratulations, you are now flying with Paparazzi!





\section*{Autonomous flight}
In this exercise you will use the provided aircraft and flight plan to avoid orange obstacles inside the CyberZoo.

\medskip
If necessary, repeat the pre-flight preparation steps listed above: charge your batteries and make sure your joystick is connected and calibrated.

\subsection*{Simulation}
It is important to test your autonomous flight plan in simulation before testing it on your real drone. This allows you to find and fix mistakes in your code before they cause damage to your drone. The simulator also allows you to prepare and test your flight plan without coming to the CyberZoo, charging batteries, etc.

\medskip
Before starting the simulator, make sure that you are no longer connected to your real drone.
Compile the autopilot by selecting the `nps' target and clicking `Clean' and `Build'. Select the `Simulation - Gazebo + Joystick' session and click `Execute'.

\medskip
For this exercise, it is convenient to place the GCS and Gazebo windows side-by-side.
The bottom-right of the GCS shows the flight plan. The flight plan consists of `blocks' of instructions, such as `Start Engine' and `Takeoff'. You can look inside a block by clicking the small triangle next to it. The currently active block is highlighted in green.

Use the flight plan to start the autonomous flight:
\begin{enumerate}
\item In the GCS, ensure the autopilot is in NAV mode.
\item Double-click the `Start Engine' block to start the engines. The throttle indicator should turn orange.
\item Double-click `Takeoff'. The drone will now take off and hover at its current position.
\item Start the orange avoider by double-clicking the `START' block. The drone will now move randomly through the CyberZoo and will yaw to avoid the orange poles.
\end{enumerate}

\medskip
If something goes wrong during autonomous flight, you may have to take over manually. This is an excellent opportunity to practice this!
Make sure that your throttle is about halfway up, then set the autopilot mode to ATT. You are now in full control of the drone. Try to land the drone in a controlled manner. Select the `Holding Point' block before switching the autopilot back to NAV mode.

\medskip
When you are finished, close the simulation using the `Stop/Remove All Processes' button.


\subsection*{OptiTrack setup}
Autonomous flight becomes significantly easier if the drone has an accurate estimate of its position and velocity. Inside the CyberZoo, an OptiTrack motion capture system is used to provide these measurements. Before your flight, you will have to add your drone in the motion tracking software. This is performed as follows:
\begin{enumerate}
\item Upload the autopilot to the Bebop. Follow the same steps as during manual flight, i.e. 1) Turn on the Bebop, press the on button four times when the light stops flashing, 2) Connect to the Bebop's WiFi access point, 3) In the Paparazzi Center, select the `ap' target and click `Clean', `Build', then `Upload'.
\item Place the drone in the center of the CyberZoo, facing the right wall as seen from the desks.
\item On the PC labeled `OPTITRACK PRIME17W PC':
\begin{enumerate}
\item Start Motive if it is not already running.
\item In the top menu, click `Layout $\rightarrow$ Capture'.
\item In the perspective view, select your drone's markers by dragging a box around them. Right click and choose `Rigid body $\rightarrow$ Create from selected markers'.
\item Under properties (bottom left), read the value of your `User ID'.
\end{enumerate}
\item On your laptop:
\begin{enumerate}
\item Plug in the ethernet cable lying on the desk. (Note: you will no longer have access to eduroam)
\item In the Paparazzi Center, start the `Flight UDP' session. Find the NatNet command textbox:
\begin{verbatim}
.../paparazzi/sw/ground_segment/misc/natnet2ivy  -ac 9999 42
\end{verbatim}
This program gets your drone's position from Motive and transmits it to the Bebop. Replace the number 9999 with your User ID number from Motive. `Stop' and `Redo' the program. You should get a message that your rigid body is now being tracked.\\
You may want to save your session with this User ID now (Session $\rightarrow$ Save). Be careful: your User ID may be different the next time you are in the CyberZoo!
\end{enumerate}
\item Ask a colleague to move your drone inside the CyberZoo. In the GCS, verify that:
\begin{itemize}
\item You have a 3D fix.
\item The position of your drone is correct: if the drone is moved to the left, it also moves to the left in the GCS.
\item The heading of the drone is correct: verify that the heading of the drone in the GCS is the same as the real drone's heading.
\item The positions of your waypoints are correct: move the drone to the waypoints in your flight plan and verify that they are inside the CyberZoo, not too close to the nets or walls and safely reachable for the drone.
\end{itemize}
\end{enumerate}



\subsection*{Autonomous flight}
At this point you have uploaded your autopilot to the Bebop and verified that it has a 3D fix.
You are now ready to start your first autonomous flight.
The steps are the same as during the simulation:
\begin{enumerate}
\item In the GCS, ensure the autopilot is in NAV mode.
\item Double-click the `Start Engine' block to start the engines. The throttle indicator should turn orange and the drone's rotors should activate.
\item Double-click `Takeoff'. The drone will now take off and hover at its current position.
\end{enumerate}
You can start the orange avoider by double-clicking the `START' block. Be ready to take over as the drone may not always see the obstacles as well as during the simulation! Also remember to watch the battery voltage as it might drop very fast.

\medskip
You can use VLC to view the drone's camera feed. Start VLC and open \verb"paparazzi/conf/video.sdp".
The Bebop's camera is rotated by 90 degrees. You can correct this in VLC by going to `Tools $\rightarrow$ Effects and filters $\rightarrow$ Video Effects $\rightarrow$ Geometry', checking `Transform' and selecting `Rotate by 270 degrees' below it.

\medskip
When you are finished, double-click on the `Land here' block to land your drone.



\section*{Next steps}
\subsection*{Edit your flightplan}
Now the power of Paparazzi is in your hands: you can now create a totally autonomous drone with the power of a flightplan. Read this page to understand what you can do: \url{http://wiki.paparazziuav.org/wiki/Flight_Plans}. 
Make sure you understand:
\begin{itemize}
	\item What a waypoint is, and what a sector is. 
	\item What exception, and while can do. 
	\item What the vertical controls are that you can use (alt, climb, throttle).
	\item What the navigation modes are that you can use (attitude, heading, go, path). 
	\item How you can set a variable and call a function. 
\end{itemize}

Try to create a simple flightplan that performs something of your choice. 

\subsection*{Create a safety rule}
Last week we discussed several problems that can make your drone crash, such as an empty battery, losing GPS or coming too close to the wall of the arena.
When flying your drone you want your drone to do something as soon as these dangerous situations occur:
\begin{itemize}
\item If your battery is empty, you want to perform a normal landing
\item If your GPS is lost, you want to perform a normal landing
\item If you fly to the wall of the arena you want to stay at the last safe point you found. 
\end{itemize}
The Paparazzi flight plan allows you to create exceptions: when the check of that exception becomes true the drone will execute a certain block. Look at the flight plan file 

\fbox{flight\_plans/tudelft/course2019\_orangeavoid\_cyberzoo.xml} to see how these exceptions are implemented. 

\clearpage
\section*{Installation manual}

This manual is only required if you're not using the supplied USB image.\\

\noindent To get the tudelft/mavlabCourse2019 branch working there are some instructions you'll have to follow. The instructions are straightforward and should be easy enough to follow. Try to understand why every step is necessary, this will improve your understanding of how everything works.

\medskip
If you have not yet installed Ubuntu, you can install it next to Windows by following this tutorial: \url{https://itsfoss.com/install-ubuntu-1404-dual-boot-mode-windows-8-81-uefi/}. We recommend you install Ubuntu 18.04 Bionic Beaver \url{https://www.ubuntu.com/download/desktop}. The rest of this manual assumes that you are running Ubuntu.

\begin{enumerate}
\item{Install Paparazzi UAV as mentioned in \url{http://wiki.paparazziuav.org/wiki/Installation}. It is recommended for Ubuntu users to use the one-liner.}
\item{Open a terminal window (Ctrl+Alt+T), navigate to the paparazzi folder and add and update the mavlabCourse remote.
\begin{lstlisting}[style=Bash]
cd paparazzi
git checkout mavlabCourse2019
git remote add mavlabCourse https://github.com/tudelft/paparazzi
git fetch mavlabCourse mavlabCourse2019
\end{lstlisting}
}
\item{Checkout the mavlabCourse2019 branch using:
\begin{lstlisting}[style=Bash]
git checkout mavlabCourse2019
\end{lstlisting}
}
\item{Sync, initialize and update the submodules using:
\begin{lstlisting}[style=Bash]
git submodule init
git submodule sync
git submodule update
\end{lstlisting}
}
\item{Build paparazzi by using:

\item{Select the right conf and control\_panel files using:
\begin{lstlisting}[style=Bash]
python start.py
\end{lstlisting}
}
\item{Install ffmpeg, vlc, cmake and java using:
\begin{lstlisting}[style=Bash]
sudo apt install ffmpeg vlc cmake default-jre
\end{lstlisting}
}
}
\item{Install and setup Gazebo simulator.
\begin{enumerate}
\item{Gazebo version 9 is recommendd with Ubuntu 18.04, earlier versions may require version 8. Install gazebo using:
\begin{lstlisting}[style=Bash]
sudo apt install gazebo9 libgazebo9-dev
\end{lstlisting}
}
\item{Open the .bashrc in your home folder
\begin{lstlisting}[style=Bash]
gedit ~/.bashrc
\end{lstlisting}
 and scroll down to the end of the file. There, add the line:\\
\verb|export GAZEBO_MODEL_PATH="<pprz>/conf/simulator/gazebo/models:$GAZEBO_MODEL_PATH"|\\
where \verb"<pprz>" should be replaced by your Paparazzi directory. Save and close the editor.}
\end{enumerate}}
\item{Install eclipse to easily navigate through the paparazzi source code.
\begin{enumerate}
\item{Download the eclipse-installer from \url{https://www.eclipse.org/downloads/download.php?file=/oomph/epp/neon/R2a/eclipse-inst-linux64.tar.gz}}
\item{Navigate to the Downloads directory, extract the .tar.gz file and run eclipse-inst.}
\item{Select the C/C++ version}
\item{It is recommended to use the default installation directory}
\item{After eclipse is installed start eclipse}
\item{Navigate to "File - New - Makefile Project with Existing Code}
\item{Name the project paparazzi select the paparazzi installation location and keep the default options}
\end{enumerate}
}
\item{Make openCV.
\begin{enumerate}
\item{Navigate to sw/ext/opencv\_bebop using "cd ./paparazzi/sw/ext/opencv\_bebop}
\item{Set up your git Email using :
\begin{lstlisting}[style=Bash]
git config --global user.email !\hl{`your\_email@host.com'}!
\end{lstlisting}}
\item{Set up your git Name using
\begin{lstlisting}[style=Bash]
git config --global user.name !\hl{`your\_name'}!}
\end{lstlisting}}
\item{Make openCV using:
\begin{lstlisting}[style=Bash]
make
\end{lstlisting}}
\end{enumerate}
}
\item{From the paparazzi folder, run start.py ``python start.py" and select\\
as Conf: \fbox{userconf/tudelft/course2019\_conf.xml}\\
and as Controlpanel: \fbox{userconf/tudelft/course2019\_control\_panel.xml}. }
\item{You can now enjoy running paparazzi and developing paparazzi in eclipse. Run paparazzi at any time by either the .desktop file located in ``./conf/system/launcher/Paparazzi.desktop" or by using ``./paparazzi" from the paparazzi folder.}
\end{enumerate}
\end{document}
